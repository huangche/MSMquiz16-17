% Type of the document
\documentclass{beamer}

% elementary packages:
\usepackage{graphicx}
\usepackage[latin1]{inputenc}
\usepackage[T1]{fontenc}
\usepackage[english]{babel}
\usepackage{listings}
\usepackage{xcolor}
\usepackage{eso-pic}
\usepackage{mathrsfs}
\usepackage{url}
\usepackage{amssymb}
\usepackage{amsmath}
\usepackage{multirow}
\usepackage{hyperref}
\usepackage{booktabs}
\usepackage{tikz}
\usepackage{ragged2e}

% additional packages
\usepackage{bbm}

% packages supplied with ise-beamer:
\usepackage{cooltooltips}
\usepackage{colordef}
\usepackage{beamerdefs}
\usepackage{lvblisting}

% Change the pictures here:
% logobig and logosmall are the internal names for the pictures: do not modify them. 
% Pictures must be supplied as JPEG, PNG or, to be preferred, PDF
\pgfdeclareimage[height=2cm]{logobig}{hulogo}
% Supply the correct logo for your class and change the file name to "logo". The logo will appear in the lower
% right corner:
\pgfdeclareimage[height=0.7cm]{logosmall}{Figures/logo.JPG}

% Title page outline:
% use this number to modify the scaling of the headline on title page
\renewcommand{\titlescale}{1.0}
% the title page has two columns, the following two values determine the percentage each one should get
\renewcommand{\titlescale}{1.0}
\renewcommand{\leftcol}{0.6}

% Define the title.Don't forget to insert an abbreviation instead 
% of "title for footer". It will appear in the lower left corner:
\title[quiz9]{Quiz 9 of Selected Topics of Mathematical Statistics Seminar}
% Define the authors:
\authora{SIVAGOUROU Dinesh} % a-b
\authorb{}
\authorc{}

% Define any internet addresses, if you want to display them on the title page:
\def\linka{http://lvb.wiwi.hu-berlin.de}
\def\linkb{}
\def\linkc{}
% Define the institute:
\institute{Ladislaus von Bortkiewicz Chair of Statistics \\
Humboldt--Universit{\"a}t zu Berlin \\}

% Comment the following command, if you don't want, that the pdf file starts in full screen mode:
%\hypersetup{pdfpagemode=FullScreen}

%Start of the document
\begin{document}

% create the title slide, layout controlled in beamerdefs.sty and the foregoing specifications
\frame[plain]{
\titlepage
}

\section{}
%1.1
%%%%%%%%%%%%%%%%%%%%%%%%%%%%%%%%%%%%%%%%%%%%%%%%%%%%%%%%%%%%%%%%%%%%%%%%%%

\frame{
\frametitle{Quiz}
\justifying

{\color{black} Let $\ X_n \xrightarrow{\ L} X $ and $ and \ Y_n  \xrightarrow{\ P} c $ where c is a finite constant . Then it holds :

\begin{itemize}
\item $ X_n + Y_n \xrightarrow{\ L} X + c $
\item $ X_nY_n \xrightarrow{\ L} Xc $
\item $ X_n/Y_n \xrightarrow{\ L} X/c $ if c $ \ne $ 0

\end{itemize}

}



\textbf{Illustrate this theorem with some examples}



}




%1.2
%%%%%%%%%%%%%%%%%%%%%%%%%%%%%%%%%%%%%%%%%%%%%%%%%%%%%%%%%%%%%%%%%%%%%%%%%%



\frame{
\frametitle{Example 1}
\justifying
Let $\ X_n \sim  B(n,n/\lambda) $ . Then it holds $\lim\limits_{n \rightarrow +\infty} X_n \xrightarrow{\ L} X $  where $ \ X \sim  P(\lambda) $ 

\

\textbf{Proof :}

$  P(X_n=x)=\dbinom{n}{x}(\frac{\lambda}{n})^x (1-\frac{\lambda}{n})^{n-x} $
\

$  =\frac{n(n-1)....(n-x+1)}{x!}(\frac{\lambda}{n})^x \frac{(1-\frac{\lambda}{n})^{n}}{(1-\frac{\lambda}{n})^{x}} $

$  =\frac{\lambda^x}{x!}(1-\frac{1}{n})...(1-\frac{x-1}{n}) \frac{(1-\frac{\lambda}{n})^{n}}{(1-\frac{\lambda}{n})^{x}} $

\

$\lim\limits_{n \rightarrow +\infty} P(X_n=x) = \frac{\lambda^x}{x!} \lim\limits_{n \rightarrow +\infty} (1-\frac{\lambda}{n})^{n} = \frac{\lambda^x}{x!}e^{-\lambda}   $

}

%1.3
%%%%%%%%%%%%%%%%%%%%%%%%%%%%%%%%%%%%%%%%%%%%%%%%%%%%%%%%%%%%%%%%%%%%%%%%%%

\frame{
\frametitle{Example 1}
\justifying
Theorem :
$\forall  c \in \mathbb{R} \ Y_n  \xrightarrow{\ P} c \Leftrightarrow \ Y_n  \xrightarrow{\ L} c	$

\

Let $\ X_n \sim  B(n,n/\lambda) $  and  $ Y_n \ defined \ by \ P(Y_n = d) = d - \frac{1}{n} $ .


$X_n \xrightarrow{\ P} X $  where $ \ X \sim  P(\lambda) $ 

$ Y_n \xrightarrow{\ P} d $


}

%1.4
%%%%%%%%%%%%%%%%%%%%%%%%%%%%%%%%%%%%%%%%%%%%%%%%%%%%%%%%%%%%%%%%%%%%%%%%%%
\frame{
\frametitle{Example 2}
\justifying

$ \sqrt[]{n} \ ( \hat{\theta} - \theta)  \sim^{L} \mathcal{N}(0,\,\sigma^{2}(\theta)) $


}

\frame{
\frametitle{Example 3}
\justifying

Let $ X_1 , X _2 , ... X_n $ a  sequence of i.i.d RV with $ E(X_i)=\mu , V(X_i)=\sigma^{2} < \infty $

\

Let  $ U_n= \sqrt[]{n} (\frac{X_n - \mu }{s_n})  $ 
\ ,
   $ C_n= \sqrt[]{n} (\frac{X_n - \mu }{\sigma})  $
\ , $ D_n= \frac{\sigma}{s_n} $
\ 

\

We have $ U_n = C_n D_n $ and since $  C_n  \xrightarrow{\ L} \mathcal{N}(0,1) $ according to Central Limit Theorem and $D_n  \xrightarrow{\ P} 1  $  according to weak law of large numbers

\

According to Slutsky's theorem we have $U_n  \xrightarrow{\ L} \mathcal{N}(0,1)  $




}



\end{document}